Purpose of the paper:
    - Using emotion elicitation alongside object detection in order to create user-profiles. The use these user-profiles to generate personalized videos that evoke emotions. 

%%%%%%%%%%%%%%%%%%%%%%
%%%% Introduction %%%%
%%%%%%%%%%%%%%%%%%%%%%
\textbf{Introduction}
 - Introduce the fact that movies are able to evoke emotions.
 - Introduce how emotions are measured
 - Briefly touch upon the possibility of combining the concept of measuring emotions and watching movies
 - Introduce emotion detection
 - Briefly introduce image detection
 - Introduce user-profiles
 - Introduce the combination of object detection and emotion detection to create user profiles
 - Introduce GANS and the possibility to generate media data.
 - End by stating what this paper will contribute
    - Emotion detection, object detection, user-profile construction and generating video

%%%%%%%%%%%%%%%%%%%%%%
%%%% Related Work %%%%
%%%%%%%%%%%%%%%%%%%%%%
\textbf{Related Work}
(Is it clear why the reader is hearing about this? ---> ):
    - Different emotions spectors: emotions are a recurring theme in the introduction
    - Measuring emotions: mentioned in the introduction
    - Detecting emotions: Mentioned in the introduction
    - User profiles : Mentioned in the introduction
    - Object recognition : Mentioned in the introduction
    - Generative Deep Learning Explained: Mentioned in the introduction

Content in section:
- Theory about Emotions
    - What emotions exists?
    - How do they correlate to a space where they can be measured?
    - What emotions do we pick?
- Theory about emotion detection
- Theory about User profiles
- Theory about object detection
- Theory about Generative Deep Learning

%%%%%%%%%%%%%%%%%%%%%%%
%%% Collecting Data? %%%
%%%%%%%%%%%%%%%%%%%%%%%
\textbf{Collecting Data}
(Is it clear why the ready is hearing about this? ---> ):
    - In the introduction, we mention emotion recognition.
    - In the introduction, we mention that we intend to build an emotion recognition model
    - In related work, we further explain emotion recognition, and the fact that it needs data.

Content in section:
- Introduce why we need a dataset
- Introduce criteria 
- How did we find the dataset
- Present datasets 
- Pick videos

%%%%%%%%%%%%%%%%%%%%%
%%%%   Study 1   %%%%
%%%%%%%%%%%%%%%%%%%%%
\textbf{Study 1}
(Is it clear why the ready is hearing about this? ---> ):
    - In the introduction, we mention emotion recognition.
    - In the introduction, we mention that we intend to build an emotion recognition model

Content in section:
- Introduction
- Participants
- Apparatus
- Tasks and procedure
- Results

%%%%%%%%%%%%%%%%%%%%%
%%%%    Model    %%%%
%%%%%%%%%%%%%%%%%%%%%
\textbf{Model}
(Is it clear why the ready is hearing about this? ---> ):
    - In the introduction, we mention emotion recognition.
    - In the introduction, we mention that we intend to build an emotion recognition model
    - In related work, we mention different methods for building emotion detection models.

Content in section:
- Why do we need a model
- What model is used
    - summary(structure)
- Hardware


