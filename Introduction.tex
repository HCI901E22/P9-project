\chapter{Introduction}
%Introduce the problem:
%Use the collection device as angle for the introduction
%Introduce ML in introduction -> new technology (ML) now allows more for personalized media
%Make it about the individual experience


%%%%%%%%%%%%%%%%
%% Attempt 1 %%
%%%%%%%%%%%%%%%
%Telltale games
Interactive experiences are increasingly becoming a part of the film industry, where interactive pieces of cinematic stories allows the user to interact with a story by making decisions that will impact the outcome of the story that is being told. One example of this is the movie on Netflix called Black Mirror: Bandersnatch, where users are presented with choices during the story where they have ten seconds to make a decision. The possibility of interaction leaves a great amount of variation in the experience that each individual is able to have. According to Netflix, there exist five different main endings and 12 endings in total for the viewer to explore, which makes it possible for users to affect the story with their opinions and emotions.
This idea of being able to use emotion in real time to optimize the experience when watching some given media data is still a new topic, and limited research has been conducted on the topic. The difficulty lies in the complex theory behind emotions and the ability to translate emotions into usable data.
Different psychological models have been introduced to attempt to map emotions to some space that is able to compare emotions in some sense. 
Combining these psychological models, biometric measurements and artificial intelligence attempts have been conducted to construct models that are able to map biometric data to specific emotions. Such a model requires biometric data that is used is able to be mapped to the different emotions within the model. Data could for instance be collected in a study, where participants are exposed to such related material while biometric data is collected. However, a lot of parameters need to be taken into consideration as they have a big impact on the quality of the data that is being collected.
\\ \\
Given the possibility that it is possible to detect emotions, such a model could be used as a live feedback loop when users are watching media clips,  in order to track their immediate emotions as they are watching media material. 
This opportunity allows for the possibility to use this information to try and make real-time changes in the video material to see if it is possible to affect what that user is feeling. With the advance in generative deep learning, such possibilities are becoming increasingly realistic to achieve. However, no prior public work has been conducted on this so far. Therefore, we aim to do so in this paper, using state-of-the-art generative deep learning in order to try and generate parts of media that will be personalized to a given user and desired emotional response. 





