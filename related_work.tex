\chapter{Related Work}
In this chapter we review related work in determining human emotional states, by using software systems. As well as, reviewing related work in generating video.
\section{Generation Video}

\section{Recognizing emotions}\label{sec:recogEmotions}
Recognizing emotions through software systems have improved in recent years. Researchers have conducted experiments measuring the emotional state of subjects, by analysing their speech \cite{SpeechEmotionRecognize}, their facial expressions \cite{FaceEmotionRecognize} or by using physiological devices measuring bio-data \cite{RecognizingEmotion, EmotionSense}. In a study by Rathod et al. \cite{BioSignalsEmotionModel} they showed 6 subjects audiovisual clips in order to induce emotions. They recorded the emotions of subjects using skin conductance and heart rate monitors, as well as recorded their facial expressions with a webcam. Their objective was for the facial recognizing and bio data monitoring to predict matching emotions. The average accuracy for neutral and happiness was 87\%, fear and sadness with 67\%, and lastly 83.13\% for anger. In another study by Khan et al. \cite{RecognizingEmotion} they implement a model that could predict 2 positive and 3 negative emotions given bio data. Their achieved accuracy ranges from 90\% to 98\%. It is hard to compare their accuracy, because their subjects, method and devices differ. However, it can be concluded that these methods used in aforementioned research can yield an accuracy that is high enough for us to take interest for our problem.
\\ \\
We will conduct experiments with the intention of collecting bio data, which we will include in a model to predict emotions. This model is then used as a tool in a video player setting, where we are interested in determining a viewers current emotional state while watching video. Beyond that we are interested in the ability to change the emotional state of a viewer, by introducing video that is supposed to elicit a specific emotion. Viewers will be wearing skin conductance and heart rate monitors as mentioned research.
%Researchers used physiological devices to determine emotions such as happy. Typical devices used are EEG \cite{EEGEmotion}, GSR \cite{GSR_Article} and PPG \cite{PPG_Article} or ECG \cite{BVP_Article}. Each of the devices have their pros and cons. For the EEG, the biggest downside is inconvenience of wearing it, however a classifier can be very accurate using EEG, up to 98.20\% in recent study \cite{EEGEmotion}. Data provided by a GSR provides a reading of the subjects' arousal, which means its close to impossible to recognize emotions solely with GSR \cite{RecognizingEmotion}. With this in mind, researchers have used GSR in combination with other devices, e.g. BVP \cite{BVP_Article} devices. Using these two bio sensors, researchers were able to achieve an accuracy of 92\% for the emotion categoires; Sad, Dislike, Joy and Stress. In other studies, researches collected data from "blood volume pulse" (BVP), "electromyogram" (EMG) \cite{EMG_Article}, “respiration”(PPG) and “skin conductance sensor”(GSR). They conducted experiments on 20 participants over 20 days, yielding a 81\% classification accuracy for 8 emotion categories (neutral, anger, hate, grief, love, romantic, joy and reverence) \cite{Experiment8EmotionCategories}.  

%\section{Generating Video}

