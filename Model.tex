\chapter{Emotion Recognition}
% Afsnit x.1 - Introduktion, da vi har dataen, kan vi lave modellen
Now that that the dataset of GSR and PPG data is available, it is possible to train a emotion recognition model. To this end, the tensorflow framework will be used to design and train the model. The model that is used for emotion detection is a CNN netowrk which is inspired from \cite{PPG_EMG_CNN}. This model consists of an input layer that takes a sample of PPG data and a sample of GST data. The hidden layers of the model, consists of 2 convolution layers, with a filter size of 32 and 64 respectively. Furthermore, both convolution layers uses a kernel size of 3x1, a stride of 1 and a ReLu activation function. Each layer is also followed by a max pooling layer, with a pool size of 2. At the end of the network, a dropout layer is used, with a dropout of 0.5, followed by a dense that serves as the output layer. The dense layer uses a filter of 4, which symbolizes low arousal, high arousal, low valance and high valance. In order to ensure that these values is a probability that sums to 1, the softmax acitvation layer is used.  

%Afsnit x.3 - Hardware


%Afsnit x.4 - Epoch, batch-size, training time

%Afsnit x.5 - Results

\begin{figure}[H]
    \centering
    \includegraphics[width=0.8\textwidth]{figures/Emotion_Detetion_Model.png}
    \caption{Emotion Detection Model}
    \label{fig:EmotionDetectionModel}
\end{figure}

In this study you will watch 16 movie clips aimed at eliciting emotion. During the study we will collect demographic information, biometric data and your responses to which emotions you experience while watching each clip. No personal data is collected and everything is handled anonymously. You are free to stop at any time. By giving consent, you acknowledge that you have read this information, voluntarily participate, and allow us to store and analyze your data from this study.