\chapter{Experiments}

\section{STUDY: COLLECT GSR AND PPG DATA USING CUSTOM DATASET}
In this study, we use the dataset collected from \cref{Datasets}, to collect GSR and PPG within the different arousal and valence levels that was chosen to represent the emotional states. The aim is to collect a dataset consisting of GSR and PPG data, which has a emotion-label within the combinations of the four-level arousal and valance model.

\subsection{Participants}
(Hopefully above 20) volunteers from the university community participated in the study. Each participant was showed 16 randomly pre-selected videos from the dataset, which means that each session took around 32 minutes to conduct. The participants used in the study, was random people within the computer science department.

\subsection{Apparatus}
Participants wore a hand-apparatus named Shimmer3 GSR+ which is able to collect GSR and PPG data, while also wearing a noise-cancellation headset.  

\subsection{Tasks and procedure}
The study was conducted in a quiet room, while wearing noise-cancellation head-phones, in order to fight any possible distracting noise. We made sure to conduct each session in the same room with the same temperature and light setting, as such parameters has an effect on the biometric data that is collected.
During a session a given participant were asked to watch a number of pre-selected  videos. For each video, the participant is instructed to sit calmly while watching the video. Meanwhile, GSR and PPG data is being collected for the participant. After the participant has watched the video, the sample of GSR and PPG data is saved, and the next video is prepared. This procedure is then repeated for each pre-selected video. 
After the participant is done watching each video, they are asked to evaluate the intensity they felt for each emotion using a 5-point Likert scale.

\subsection{Results}
Conducting this study, a dataset was constructed consisting of samples of GSR and PPG data that is labeled with a valance and a arousal value.