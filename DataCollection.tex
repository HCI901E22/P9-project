%Afsnit x.1 - Forklaring af emotion detection model, samt hvilket data der skal bruges. 
In order to be able to infer emotion from the GSR and PPG data, it is necessary to have a model that is able to map the data to an actual feeling accurately. To this end, an emotion detection model will be constructed, that is able to do this mapping. Such a model would take in a sample of GSR and PPG data, which it is then able to map into an emotion. In order to train such a model, a dataset would be required that consists of samples of PPG and GSR data that is mapped to a specific emotion. 

%Afsnit x.2: Data collection - Udforskning af eksisterende dataset & Konstruktion af dataset til at lave emotion detection model, samt interviews til at validere kategorien af dataen 
Such a dataset can either be built manually or found on the internet. Due to time-constraints, it would be preferable to find such a dataset online. Therefore, effort was put into finding such a dataset. However while different datasets exist on the internet, these where either not compatible with our data-requirements or private. Therefore, it is necessary to construct a dataset that consists of GSR and PPG samples along with an emotion label. In order to construct this dataset, a choice was made to conduct a study where a total of xx participants will be presented images/videos within a given emotion category while GSR and PPG data is being collected. However, before this study can be conducted, a collection of images/videos within different emotion categories will need to be constructed. 

% DATA COLLECTION STUDY %
In order to gather this collection of images, an initial collection of images will be gathered from the FER-2013\cite{dataset} dataset. This dataset contains 28000 labeled images, where each image is labeled with one of seven emotions. The images are gathered from google searches of each emotions and synonyms of the emotions. From this dataset, 5 images for each emotional state is picked that is deemed best to represent the given emotional state.
A total of 16 different emotional states will be used, where each state is represented through the 4-level valance and arousal score. This score is inspired from \cite{CNNEmotionDetection}, and provides a more detailed mapping of emotions to the normal 2-level approach.
This leaves the initial collection of images with a total of 80 images, where there exist 5 images for each emotional state.
In order to reduce this dataset, to be a more compact collection of images that is also more representative for each emotion state, a study will be conducted.
The study includes xx participants that will be presented with the selection of images/videos where the participant will see one category of images/videos at a time. After seeing the images for a given category, the user will be asked to pick out 2 images/videos that they think best represent the emotion category. After the study, the two images from each category that scored the highest score will be chosen to stay while the rest is discarded, which leaves a total of 32 images/videos in total. This dataset will then be used to conduct the study to collect the GSR and PPG data. 

% ACTUAL STUDY AND POST PROCESSING
For the study to collect GSR and PPG data, a total of xx participants are participating. Each participant is presented with an image/video from the dataset at a time while GST and PPG data are being measured. After the participant has finished the measuring stage, they will be presented with a questionnaire to each image, where the participant is aksed what they felt when they were shown a given video/image along with the possibility to elaborate. 
After the study is complete, the data from the study is post-processed by going through the interaction between media and participants. Here, the survey and the media emotion category is being cross-checked to ensure that the dataset will not contain any misleading data. Furthermore, the data is being split into samples of data, which will contains 2000 stamps of GSR data, 2000 stamps of PPG data and a label. 

%Afsnit x.5 - Endelige dataset
The study left us with a total of xx samples of GSR- and PPG-data spanning across 16 different emotional states. 